% Options
\PassOptionsToPackage{unicode}{hyperref}
\PassOptionsToPackage{hyphens}{url}

\documentclass[]{article}

\usepackage{amsmath,amssymb}
\usepackage{lmodern}
\usepackage{iftex}
\ifPDFTeX
  \usepackage[T1]{fontenc}
  \usepackage[utf8]{inputenc}
  \usepackage{textcomp}
\else
  \usepackage{unicode-math}
\fi

\IfFileExists{microtype.sty}{\usepackage[]{microtype}}{}
\IfFileExists{parskip.sty}{\usepackage{parskip}}{
  \setlength{\parindent}{0pt}
  \setlength{\parskip}{6pt plus 2pt minus 1pt}
}

\usepackage{xcolor}
\usepackage{graphicx}
\usepackage{longtable,booktabs,array}
\usepackage{calc}
\usepackage{etoolbox}
\usepackage{footnote}
\usepackage{listings}

\makesavenoteenv{longtable}

\lstset{
  language=C,
  basicstyle=\ttfamily\small,
  commentstyle=\itshape\color{gray},
  keywordstyle=\color{blue}\bfseries,
  stringstyle=\color{red},
  numbers=left,
  numberstyle=\tiny,
  breaklines=true,
  frame=single,
  tabsize=2
}

\setlength{\emergencystretch}{3em}

\usepackage{hyperref}
\hypersetup{hidelinks}

\begin{document}

%%%%%%%%%%%%%%%%%%%%%%% CAPA %%%%%%%%%%%%%%%%%%%%%%%%%%%%%

\begin{titlepage}
    \centering

    {\Large \textbf{Universidade de Aveiro}}\\[4pt]
    {\large Departamento de Eletrónica, Telecomunicações e Informática}\\[30pt]

    {\Large \textbf{Algoritmos e Estruturas de Dados}}\\[40pt]

    {\Huge \textbf{Relatório do Trabalho 1}}\\[10pt]
    {\LARGE \textbf{Análise de Algoritmos sobre Imagens RGB}}\\[40pt]

    \begin{flushleft}
    \textbf{Autores:}\\[4pt]
    Vicente Amorim Silva – 125160\\
    vicenteamorimsilva@ua.pt\\
    Victor Miguel L. V. da Costa Morais – 125478\\
    victorcostamorais@ua.pt\\[18pt]
    \end{flushleft}

    \vfill
    \begin{center}
        \includegraphics[width=5cm]{ua.pdf}
    \end{center}
    \vfill

    {\Large \textbf{Universidade de Aveiro – DETI}}
\end{titlepage}

%%%%%%%%%%%%%%%%%%%%%%% INTRODUÇÃO %%%%%%%%%%%%%%%%%%%%%%%%%%%%%

\section*{Introdução}

O presente relatório descreve a análise formal e experimental de funções de processamento de imagens desenvolvidas no âmbito do módulo \textbf{imageRGB} da unidade curricular \emph{Algoritmos e Estruturas de Dados}. O módulo recorre a uma representação indexada composta por uma matriz de labels e uma tabela de cores (LUT), permitindo operações eficientes de manipulação e comparação de imagens RGB.

O trabalho inclui:  
(1) a caracterização formal da função \textbf{ImageIsEqual};  
(2) a avaliação experimental do comportamento desta função em vários cenários;  
(3) a comparação entre diferentes estratégias de segmentação de regiões, fundamentais em algoritmos de \emph{region growing}.  

O objetivo final consiste em avaliar o desempenho e a adequação destas abordagens para imagens indexadas, apoiando-se em análise teórica e evidência experimental.

%%%%%%%%%%%%%%%%%%%%%%% SECÇÃO 1 %%%%%%%%%%%%%%%%%%%%%%%%%%%%%

\section*{1. Estrutura Interna do TAD imageRGB}

O módulo \textbf{imageRGB} representa imagens através de uma matriz de labels inteiros, onde cada label corresponde a uma cor armazenada numa LUT (look-up table) de valores RGB de 24 bits. Esta abordagem reduz redundância, facilita a partilha de cores e melhora o desempenho de operações repetidas.

\textbf{A estrutura contém:}
\begin{itemize}
    \item \textbf{width} – número de colunas da imagem
    \item \textbf{height} – número de linhas
    \item \textbf{image} – matriz com labels inteiros
    \item \textbf{LUT} – vetor com cores RGB reais
    \item \textbf{num\_colors} – número de cores utilizadas
\end{itemize}

Esta estrutura é eficiente em termos de memória e garante uma separação clara entre representação lógica (labels) e representação física da cor (LUT).

%%%%%%%%%%%%%%%%%%%%%%% SECÇÃO 2 %%%%%%%%%%%%%%%%%%%%%%%%%%%%%

\section*{2. Análise Formal da Função ImageIsEqual}

A função \textbf{ImageIsEqual} verifica se duas imagens são visualmente idênticas.  
Primeiro confirma igualdade de dimensões; se forem iguais, percorre pixel a pixel, obtendo a cor real através da LUT e comparando valores RGB.

\newpage
\subsection*{2.1 Melhor Caso — $\Omega(1)$}

O melhor caso ocorre quando:
\begin{itemize}
  \item as dimensões das imagens diferem; ou
  \item o primeiro píxel apresenta cores distintas.
\end{itemize}

Ambos os cenários exigem apenas uma verificação inicial, resultando em custo constante.

\[
T_{\text{melhor}}(N) = \Omega(1)
\]

\subsection*{2.2 Pior Caso — $\mathcal{O}(W \times H)$}

O pior caso ocorre quando todas as cores coincidem. A função percorre todos os píxeis, realizando uma comparação por píxel.  

\[
T_{\text{pior}}(N) \in \mathcal{O}(N)
\]

\subsection*{2.3 Caso Médio — $\Theta(W \times H)$}

Assumindo diferenças distribuídas de forma uniforme, a divergência ocorre em média a meio da imagem. A função executa aproximadamente $N/2$ comparações de cor.

\[
T_{\text{médio}}(N) \in \Theta(N)
\]

\subsection*{2.4 Complexidade Espacial}

A função utiliza apenas um conjunto fixo de variáveis auxiliares.

\[
S(N) \in \mathcal{O}(1)
\]

\subsection*{2.5 Independência da LUT}

A função compara sempre cores reais e não labels; logo, imagens são consideradas iguais independentemente da organização interna da LUT, desde que os valores RGB correspondam.

%%%%%%%%%%%%%%%%%%%%%%% SECÇÃO 3 %%%%%%%%%%%%%%%%%%%%%%%%%%%%%
\newpage
\section*{3. Metodologia Experimental}

Os testes foram realizados utilizando imagens sintéticas de diferentes dimensões e padrões de variação, de forma a analisar o comportamento da função \textbf{ImageIsEqual} em múltiplos cenários.  
Foram consideradas imagens idênticas, imagens com diferenças localizadas, imagens aleatórias, imagens com ruído e versões com LUT diferente mas com cores equivalentes.

Para cada teste registou-se o número de comparações de cor até a função terminar.

%%%%%%%%%%%%%%%%%%%%%%% SECÇÃO 4 %%%%%%%%%%%%%%%%%%%%%%%%%%%%%

\section*{4. Resultados Experimentais}

\subsection*{4.1 Tabela para cenários principais}

\begin{longtable}[]{@{}lllll@{}}
\toprule
\textbf{Tamanho} & \textbf{Pixels} & \textbf{Iguais} & \textbf{Diferença cedo} & \textbf{Caso médio}\\
\midrule
\endhead
50×50 & 2 500 & 2 500 & 1 & 830\\
100×100 & 10 000 & 10 000 & 1 & 3 300\\
150×150 & 22 500 & 22 500 & 1 & 7 500\\
200×200 & 40 000 & 40 000 & 1 & 13 200\\
\bottomrule
\end{longtable}

\subsection*{4.2 Cenários adicionais}

\begin{longtable}[]{@{}llllll@{}}
\toprule
\textbf{Tamanho} & \textbf{Dim. Dif.} & \textbf{LUT Dif.} & \textbf{Ruído 1\%} & \textbf{Últimos píxeis} & \textbf{Aleatória}\\
\midrule
\endhead
50×50 & 0 comps & 2500 & ~25 & ~2490 & 1–3\\
100×100 & 0 comps & 10000 & ~100 & ~9990 & 1–4\\
200×200 & 0 comps & 40000 & ~400 & ~39990 & 1–5\\
\bottomrule
\end{longtable}

\subsection*{4.3 Análise detalhada dos cenários}

Os resultados obtidos podem ser divididos em quatro categorias principais, cada uma revelando um comportamento diferente da função \textbf{ImageIsEqual}:

\subsubsection*{Cenários de tempo constante ($\Omega(1)$)}
Estes correspondem aos casos em que a função termina imediatamente:
\begin{itemize}
    \item dimensões diferentes;
    \item diferença logo no primeiro píxel;
    \item imagens totalmente aleatórias (quase sempre).
\end{itemize}
Nestes casos a função não depende do tamanho da imagem, uma vez que a condição de término é satisfeita antes de percorrer a matriz.

\subsubsection*{Cenários lineares ($\Theta(N)$)}
Incluem:
\begin{itemize}
    \item caso médio;
    \item ruído esparso;
    \item regiões homogéneas com diferenças aleatórias;
    \item imagens iguais (pior caso).
\end{itemize}
Nestes casos a função percorre um número significativo de píxeis até encontrar uma diferença (ou não encontrar nenhuma). O custo cresce proporcionalmente ao número total de elementos.

\subsubsection*{Pior caso absoluto}
Ocorre quando:
\begin{itemize}
    \item as imagens são idênticas em todas as posições.
\end{itemize}
A função realiza exatamente $N$ comparações, validando o limite superior teórico.

\subsubsection*{Interpretação global}
Os resultados confirmam de forma consistente a análise formal:
\begin{itemize}
    \item a função apresenta comportamento predominantemente linear;
    \item existe uma transição clara entre os cenários de término precoce e os cenários de varrimento completo;
    \item imagens reais, que raramente são totalmente iguais, tendem a aproximar-se do caso médio.
\end{itemize}
Em síntese, a função \textbf{ImageIsEqual} é eficiente, estável e previsível, apresentando desempenho adequado para comparação de imagens indexadas.

%%%%%%%%%%%%%%%%%%%%%%% SECÇÃO 5 %%%%%%%%%%%%%%%%%%%%%%%%%%%%%
\newpage
\section*{5. Comparação das Estratégias de Region Growing}

No contexto da segmentação de regiões, foram implementadas e avaliadas três estratégias de \textit{region growing}, analisando-se a sua eficiência e robustez em diferentes tipos de imagens.

\textbf{As abordagens estudadas foram:}
\begin{itemize}
    \item \textbf{Recursiva} — simples, mas limitada pela profundidade da pilha;
    \item \textbf{Stack (DFS)} — versão iterativa que elimina o risco de \emph{stack overflow};
    \item \textbf{Queue (BFS)} — propagação equilibrada e radial, adequada a regiões extensas.
\end{itemize}


\subsection*{5.1 Desempenho Comparativo}

\begin{longtable}[]{@{}llll@{}}
\toprule
\textbf{Estratégia} & \textbf{Região compacta} & \textbf{Labirinto} & \textbf{Regiões pequenas}\\
\midrule
\endhead
Recursiva & Boa até 180×180 & Falha (stack) & Excelente\\
Stack (DFS) & Ótima & Excelente & Muito boa\\
Queue (BFS) & Boa & Mais lenta & Boa\\
\bottomrule
\end{longtable}

A comparação evidencia diferenças claras: a versão recursiva apresenta riscos de falha, a variante com pilha é rapidamente estável em cenários complexos e a versão com fila assegura uma expansão mais regular em imagens largas.

%%%%%%%%%%%%%%%%%%%%%%% CONCLUSÃO %%%%%%%%%%%%%%%%%%%%%%%%%%%%%

\section*{Conclusão}

A análise realizada demonstra que a representação indexada utilizada no módulo \textbf{imageRGB} permite operações eficientes de comparação e segmentação.  
A função \textbf{ImageIsEqual} apresenta um comportamento determinístico, com complexidade linear no pior e no caso médio, validada experimentalmente.  
A comparação entre as estratégias de segmentação mostra que abordagens iterativas são preferíveis para imagens de maior dimensão, garantindo robustez e estabilidade.

Os resultados obtidos confirmam a adequação das estruturas e algoritmos estudados, permitindo uma compreensão sólida dos princípios fundamentais de processamento de imagens em contexto de estruturas de dados.

\end{document}
