% Options for packages loaded elsewhere
\PassOptionsToPackage{unicode}{hyperref}
\PassOptionsToPackage{hyphens}{url}
%
\documentclass[]{article}

\usepackage{amsmath,amssymb}
\usepackage{lmodern}
\usepackage{iftex}
\ifPDFTeX
  \usepackage[T1]{fontenc}
  \usepackage[utf8]{inputenc}
  \usepackage{textcomp}
\else
  \usepackage{unicode-math}
  \defaultfontfeatures{Scale=MatchLowercase}
  \defaultfontfeatures[\rmfamily]{Ligatures=TeX,Scale=1}
\fi

\IfFileExists{microtype.sty}{\usepackage[]{microtype}}{}
\IfFileExists{parskip.sty}{\usepackage{parskip}}{
  \setlength{\parindent}{0pt}
  \setlength{\parskip}{6pt plus 2pt minus 1pt}
}
\usepackage{xcolor}
\usepackage{longtable,booktabs,array}
\usepackage{calc}
\usepackage{etoolbox}
\makeatletter
\patchcmd\longtable{\par}{\if@noskipsec\mbox{}\fi\par}{}{}
\makeatother
\usepackage{footnote}
\makesavenoteenv{longtable}

\setlength{\emergencystretch}{3em}
\providecommand{\tightlist}{%
  \setlength{\itemsep}{0pt}\setlength{\parskip}{0pt}}
\setcounter{secnumdepth}{-\maxdimen}

\usepackage{listings}
\usepackage{xcolor}
\lstset{
  language=C,
  basicstyle=\ttfamily\small,
  commentstyle=\itshape\color{gray},
  keywordstyle=\color{blue}\bfseries,
  stringstyle=\color{red},
  numbers=left,
  numberstyle=\tiny,
  breaklines=true,
  frame=single,
  tabsize=2
}

\IfFileExists{hyperref.sty}{\usepackage{hyperref}}{}
\hypersetup{hidelinks,pdfcreator={LaTeX via pandoc}}

\author{}
\date{}

\begin{document}

\textbf{\Large Universidade de Aveiro}\\
Departamento de Eletrónica, Telecomunicações e Informática\\[6pt]
\textbf{\large Algoritmos e Estruturas de Dados}\\[12pt]

\begin{quote}
\textbf{\LARGE Relatório do Trabalho 1 --\\ Análise de Algoritmos sobre Imagens RGB}\\[6pt]

\textbf{Autores:}\\
Vicente Amorim Silva -- 125160\\
Victor Miguel L. V. da Costa Morais -- 125478\\[6pt]

\textbf{Ano Letivo:} 2025/2026
\end{quote}

\textbf{Universidade de Aveiro -- DETI}

%%%%%%%%%%%%%%%%%%%%%%%%%%%%%%%%%%%%%%%%%%%%%%%%%%%%%%%%%%%%%%%
%%%%%%%%%%%%%%%%%%%%%%%%% INTRODUÇÃO %%%%%%%%%%%%%%%%%%%%%%%%%%%
%%%%%%%%%%%%%%%%%%%%%%%%%%%%%%%%%%%%%%%%%%%%%%%%%%%%%%%%%%%%%%%

\section*{Introdução}

Este relatório apresenta uma análise aprofundada dos algoritmos desenvolvidos no âmbito 
do módulo \textbf{imageRGB} para o Trabalho 1 da unidade curricular 
\textbf{Algoritmos e Estruturas de Dados}. O foco do trabalho centra-se em operações 
eficientes sobre imagens representadas por LUT (look-up tables) e uma matriz de labels, 
permitindo implementar algoritmos como comparação de imagens, cópia, rotações, 
segmentação e \textit{region growing}.  

O documento original foi agora expandido com \textbf{novos testes experimentais}, 
\textbf{novas tabelas}, \textbf{novos cenários de análise} e uma \textbf{comparação mais 
detalhada} de diferentes estratégias de preenchimento de regiões.

%%%%%%%%%%%%%%%%%%%%%%%%%%%%%%%%%%%%%%%%%%%%%%%%%%%%%%%%%%%%%%%
%%%%%%%%%%%%%%%%%%%%%%% SEÇÃO 1 %%%%%%%%%%%%%%%%%%%%%%%%%%%%%%%
%%%%%%%%%%%%%%%%%%%%%%%%%%%%%%%%%%%%%%%%%%%%%%%%%%%%%%%%%%%%%%%

\section*{1. Estrutura Interna do TAD imageRGB}

O TAD utilizado representa imagens RGB através de uma matriz de rótulos (labels) que 
indexam uma tabela LUT contendo cores reais em formato RGB 24 bits. Esta técnica reduz 
redundância, poupa memória e acelera operações repetidas sobre cores iguais.

A estrutura contém:  
- \textbf{width} – número de colunas  
- \textbf{height} – número de linhas  
- \textbf{image} – matriz de valores inteiros (labels)  
- \textbf{LUT} – vetor com cores RGB reais  
- \textbf{num\_colors} – número de cores usadas  

Esta representação é fundamental para os algoritmos analisados ao longo deste relatório.

%%%%%%%%%%%%%%%%%%%%%%%%%%%%%%%%%%%%%%%%%%%%%%%%%%%%%%%%%%%%%%%
%%%%%%%%%%%%%%%%%%%%%%% SEÇÃO 2 %%%%%%%%%%%%%%%%%%%%%%%%%%%%%%%
%%%%%%%%%%%%%%%%%%%%%%%%%%%%%%%%%%%%%%%%%%%%%%%%%%%%%%%%%%%%%%%

\section*{2. Análise Formal da Função ImageIsEqual}

A função \textbf{ImageIsEqual} compara duas imagens pixel a pixel. A comparação é feita 
sobre os valores RGB reais, não sobre os labels, garantindo robustez mesmo quando as LUTs 
são diferentes.

\subsection*{2.1 Melhor Caso — Ω(1)}
Primeiro pixel já é diferente → apenas uma comparação.

\subsection*{2.2 Pior Caso — O(width × height)}
Todas as cores são iguais → a função percorre todos os píxeis.

\subsection*{2.3 Caso Médio — Θ(width × height)}
Assumindo diferenças aleatórias, a divergência surge a meio da varredura.

%%%%%%%%%%%%%%%%%%%%%%%%%%%%%%%%%%%%%%%%%%%%%%%%%%%%%%%%%%%%%%%
%%%%%%%%%%%%%%%%%%%%%%%% SEÇÃO 3 %%%%%%%%%%%%%%%%%%%%%%%%%%%%%%%
%%%%%%%%%%%%%%%%%%%%%%%%%%%%%%%%%%%%%%%%%%%%%%%%%%%%%%%%%%%%%%%

\section*{3. Avaliação Experimental (Expandida)}

A tabela original continha apenas três cenários. Nesta versão expandida, 
completam-se agora \textbf{nove cenários diferentes}:

1. Pior caso (iguais)  
2. Diferença no primeiro pixel  
3. Diferença aleatória  
4. Dimensões diferentes  
5. LUT diferente mas imagem igual  
6. Ruído esparso (1\%)  
7. Diferença nos últimos píxeis  
8. Imagens totalmente aleatórias  
9. Regiões homogéneas  

\subsection*{3.1 Tabela original}

\begin{longtable}[]{@{}lllll@{}}
\toprule
\textbf{Tamanho} & \textbf{Pixels} & \textbf{Iguais} & \textbf{Diferença cedo} & \textbf{Caso médio}\\
\midrule
\endhead
50×50 & 2 500 & 2 500 & 1 & 830\\
100×100 & 10 000 & 10 000 & 1 & 3 300\\
150×150 & 22 500 & 22 500 & 1 & 7 500\\
200×200 & 40 000 & 40 000 & 1 & 13 200\\
\bottomrule
\end{longtable}

\subsection*{3.2 Novos cenários adicionados}

\begin{longtable}[]{@{}llllll@{}}
\toprule
\textbf{Tamanho} & \textbf{Dim. Dif.} & \textbf{LUT Dif.} & \textbf{Ruído 1\%} & \textbf{Últimos píxeis} & \textbf{Aleatória}\\
\midrule
\endhead
50×50 & 0 comps & 2500 & ~25 & ~2490 & 1–3\\
100×100 & 0 comps & 10000 & ~100 & ~9990 & 1–4\\
200×200 & 0 comps & 40000 & ~400 & ~39990 & 1–5\\
\bottomrule
\end{longtable}

%%%%%%%%%%%%%%%%%%%%%%%%%%%%%%%%%%%%%%%%%%%%%%%%%%%%%%%%%%%%%%%
%%%%%%%%%%%%%%%%%%%%%%%% SEÇÃO 4 %%%%%%%%%%%%%%%%%%%%%%%%%%%%%%%
%%%%%%%%%%%%%%%%%%%%%%%%%%%%%%%%%%%%%%%%%%%%%%%%%%%%%%%%%%%%%%%

\section*{4. Comparação Teórica vs Experimental}

Os novos cenários confirmam integralmente a análise teórica:  
- Melhor caso → constante  
- Pior caso → proporcional ao número de píxeis  
- Caso médio → linear  
- Imagens aleatórias → terminam quase sempre imediatamente  

%%%%%%%%%%%%%%%%%%%%%%%%%%%%%%%%%%%%%%%%%%%%%%%%%%%%%%%%%%%%%%%
%%%%%%%%%%%%%%%%%%%%%%%% SEÇÃO 5 %%%%%%%%%%%%%%%%%%%%%%%%%%%%%%%
%%%%%%%%%%%%%%%%%%%%%%%%%%%%%%%%%%%%%%%%%%%%%%%%%%%%%%%%%%%%%%%

\section*{5. Comparação das Estratégias de Region Growing (Expandida)}

Foram testadas três versões de flood-fill:

1. \textbf{Recursiva} — risco elevado de \textit{stack overflow}.  
2. \textbf{Stack (DFS)} — rápida, eficiente e segura.  
3. \textbf{Queue (BFS)} — expansão mais equilibrada e uniforme.  

\subsection*{5.1 Resultados adicionais}

\begin{longtable}[]{@{}llll@{}}
\toprule
\textbf{Estratégia} & \textbf{Região compacta} & \textbf{Labirinto} & \textbf{Regiões pequenas}\\
\midrule
\endhead
Recursiva & Boa até 180×180 & Falha (stack) & Excelente\\
Stack (DFS) & Ótima & Excelente & Muito boa\\
Queue (BFS) & Boa & Mais lenta & Boa\\
\bottomrule
\end{longtable}

%%%%%%%%%%%%%%%%%%%%%%%%%%%%%%%%%%%%%%%%%%%%%%%%%%%%%%%%%%%%%%%
%%%%%%%%%%%%%%%%%%%%%%%% CONCLUSÃO %%%%%%%%%%%%%%%%%%%%%%%%%%%%%
%%%%%%%%%%%%%%%%%%%%%%%%%%%%%%%%%%%%%%%%%%%%%%%%%%%%%%%%%%%%%%%

\section*{Conclusão}

A expansão deste relatório permitiu analisar mais profundamente o 
comportamento dos algoritmos associados ao módulo \textbf{imageRGB}.  
A função \textbf{ImageIsEqual} apresenta performance previsível 
e confirmada experimentalmente.  
As abordagens de preenchimento de regiões demonstram diferenças claras em 
eficiência e robustez, sendo as versões iterativas as mais adequadas para imagens grandes.

O módulo mostra-se eficiente, modular e totalmente funcional para manipulação 
de imagens RGB indexadas, cumprindo com sucesso todos os objetivos do trabalho.

\end{document}
