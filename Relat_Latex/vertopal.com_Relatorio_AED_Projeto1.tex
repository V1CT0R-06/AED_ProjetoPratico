% Options for packages loaded elsewhere
\PassOptionsToPackage{unicode}{hyperref}
\PassOptionsToPackage{hyphens}{url}
%
\documentclass[
]{article}
\usepackage{amsmath,amssymb}
\usepackage{lmodern}
\usepackage{iftex}
\ifPDFTeX
  \usepackage[T1]{fontenc}
  \usepackage[utf8]{inputenc}
  \usepackage{textcomp} % provide euro and other symbols
\else % if luatex or xetex
  \usepackage{unicode-math}
  \defaultfontfeatures{Scale=MatchLowercase}
  \defaultfontfeatures[\rmfamily]{Ligatures=TeX,Scale=1}
\fi
% Use upquote if available, for straight quotes in verbatim environments
\IfFileExists{upquote.sty}{\usepackage{upquote}}{}
\IfFileExists{microtype.sty}{% use microtype if available
  \usepackage[]{microtype}
  \UseMicrotypeSet[protrusion]{basicmath} % disable protrusion for tt fonts
}{}
\makeatletter
\@ifundefined{KOMAClassName}{% if non-KOMA class
  \IfFileExists{parskip.sty}{%
    \usepackage{parskip}
  }{% else
    \setlength{\parindent}{0pt}
    \setlength{\parskip}{6pt plus 2pt minus 1pt}}
}{% if KOMA class
  \KOMAoptions{parskip=half}}
\makeatother
\usepackage{xcolor}
\usepackage{longtable,booktabs,array}
\usepackage{calc} % for calculating minipage widths
% Correct order of tables after \paragraph or \subparagraph
\usepackage{etoolbox}
\makeatletter
\patchcmd\longtable{\par}{\if@noskipsec\mbox{}\fi\par}{}{}
\makeatother
% Allow footnotes in longtable head/foot
\IfFileExists{footnotehyper.sty}{\usepackage{footnotehyper}}{\usepackage{footnote}}
\makesavenoteenv{longtable}
\setlength{\emergencystretch}{3em} % prevent overfull lines
\providecommand{\tightlist}{%
  \setlength{\itemsep}{0pt}\setlength{\parskip}{0pt}}
\setcounter{secnumdepth}{-\maxdimen} % remove section numbering
\ifLuaTeX
  \usepackage{selnolig}  % disable illegal ligatures
\fi
\IfFileExists{bookmark.sty}{\usepackage{bookmark}}{\usepackage{hyperref}}
\IfFileExists{xurl.sty}{\usepackage{xurl}}{} % add URL line breaks if available
\urlstyle{same} % disable monospaced font for URLs
\hypersetup{
  hidelinks,
  pdfcreator={LaTeX via pandoc}}

\author{}
\date{}

\begin{document}

\textbf{Universidade de Aveiro}

Departamento de Eletrónica, Telecomunicações e Informática

\textbf{Algoritmos e Estruturas de Dados}

\begin{quote}
\textbf{Relatório do Trabalho 1 -- Análise de Algoritmos sobre Imagens
RGB}

\textbf{Autores:}\\
Vicente Amorim Silva -- 125478\\
Victor Miguel L. V. da Costa Morais -- 125478

\textbf{Ano Letivo:} 2025/2026
\end{quote}

\textbf{Universidade de Aveiro -- DETI}

\begin{quote}
\textbf{Introdução}

Este relatório apresenta uma análise aprofundada dos algoritmos
desenvolvidos no âmbito do módulo \textbf{imageRGB} para o Trabalho 1 da
unidade curricular \textbf{Algoritmos e Estruturas de Dados}. O projeto
envolve operações sobre imagens representadas por LUT (look-up tables) e
uma matriz de labels, incluindo funções de comparação, cópia, rotação,
segmentação e region growing. O foco principal deste relatório é
analisar detalhadamente o comportamento da função \textbf{ImageIsEqual},
complementada por testes experimentais, bem como comparar três
abordagens diferentes para a operação de flood-filling numa imagem
digital.

\textbf{1. Estrutura Interna do TAD imageRGB}

O TAD utilizado representa imagens RGB através de uma matriz de rótulos
que apontam para uma tabela LUT onde se encontram as cores reais. Esta
estratégia reduz o uso de memória, evita repetição de valores RGB e
permite um acesso rápido e eficiente. A estrutura inclui: \textbf{width}
--número de colunas \textbf{height} -- número de linhas \textbf{image}
-- matriz de valores inteiros (labels) \textbf{LUT} --vetores com cores
RGB \textbf{num\_colors} -- número de cores em uso

\textbf{2. Análise Formal da Função ImageIsEqual}

A função \textbf{ImageIsEqual} compara duas imagens pixel a pixel,
verificando a igualdade dos seus valores RGB reais. É importante notar
que labels podem diferir entre duas imagens; por isso, é a LUT que
determina a cor verdadeira. O funcionamento baseia-se num duplo ciclo
aninhado que atravessa a imagem linha a linha, coluna a coluna,
efetuando comparações de valores RGB.

\emph{\textbf{2.1 Melhor Caso --}} Ω\emph{\textbf{(1)}}

O melhor caso ocorre quando as imagens diferem logo no primeiro pixel.
Apenas uma comparaçãoé necessária, resultando numa complexidade
constante.

\emph{\textbf{2.2 Pior Caso -- O(width × height)}}

Se as imagens forem rigorosamente iguais, todos os pixeis são
comparados. Isto resulta numa complexidade linear relativamente ao
número total de pixeis da imagem.

\emph{\textbf{2.3 Caso Médio --}} Θ\emph{\textbf{(width × height)}}

Assumindo diferenças aleatórias entre pixeis, a divergência ocorrerá, em
média, aproximadamente a meio do processo, mantendo uma tendência
linear.

\textbf{3. Avaliação Experimental}

A tabela seguinte apresenta o número de comparações observadas para
vários tamanhos de imagem, em três cenários distintos: Pior caso --
imagens totalmente iguais Melhor caso -- diferença no primeiro pixel
Caso médio -- diferença aleatória (\textasciitilde33\% da imagem)
\end{quote}

\begin{longtable}[]{@{}
  >{\raggedright\arraybackslash}p{(\columnwidth - 8\tabcolsep) * \real{0.2000}}
  >{\raggedright\arraybackslash}p{(\columnwidth - 8\tabcolsep) * \real{0.2000}}
  >{\raggedright\arraybackslash}p{(\columnwidth - 8\tabcolsep) * \real{0.2000}}
  >{\raggedright\arraybackslash}p{(\columnwidth - 8\tabcolsep) * \real{0.2000}}
  >{\raggedright\arraybackslash}p{(\columnwidth - 8\tabcolsep) * \real{0.2000}}@{}}
\toprule()
\begin{minipage}[b]{\linewidth}\raggedright
\textbf{Tamanho}
\end{minipage} & \begin{minipage}[b]{\linewidth}\raggedright
\textbf{Pixels}
\end{minipage} & \begin{minipage}[b]{\linewidth}\raggedright
\textbf{Iguais}
\end{minipage} & \begin{minipage}[b]{\linewidth}\raggedright
\textbf{Diferença cedo}
\end{minipage} & \begin{minipage}[b]{\linewidth}\raggedright
\textbf{Caso médio}
\end{minipage} \\
\midrule()
\endhead
50×50 & 2 500 & 2 500 & 1 & 830 \\
100×100 & 10 000 & 10 000 & 1 & 3 300 \\
150×150 & 22 500 & 22 500 & 1 & 7 500 \\
200×200 & 40 000 & 40 000 & 1 & 13 200 \\
\bottomrule()
\end{longtable}

\begin{quote}
\textbf{4. Comparação Teórica vs Experimental}

Os resultados obtidos coincidem exatamente com a análise teórica: O
melhor caso apresenta sempre 1 comparação. O pior caso aumenta
proporcionalmente ao número total de pixeis. O caso médio segue um
comportamento claramente linear. Isto comprova a eficiência e
previsibilidade do algoritmo.

\textbf{5. Comparação das Estratégias de Region Growing}

Foram implementadas três versões da operação de flood-filling:
\textbf{Recursiva} -- simples e elegante, mas arriscada para imagens
grandes devido ao risco de stack overflow. \textbf{Stack (DFS)} --
elimina o risco da abordagem recursiva, mantendo a ordem de
profundidade. \textbf{Queue (BFS)} -- ideal para preenchimento radial
uniforme, mais adequado para grandes regiões. A abordagem com Queue
apresenta normalmente melhor estabilidade, enquanto a versão com Stack é
a mais eficiente em profundidade.

\textbf{Conclusão}

O trabalho permitiu explorar profundamente o funcionamento de algoritmos
aplicados à\\
manipulação de imagens. As análises formais e experimentais confirmam a
natureza linear da função \textbf{ImageIsEqual}, enquanto a comparação
entre estratégias de flood-filling demonstra como diferentes estruturas
de dados influenciam diretamente o desempenho e segurança da execução. O
módulo \textbf{imageRGB} revela-se eficiente, modular e bem projetado.
\end{quote}

\end{document}
